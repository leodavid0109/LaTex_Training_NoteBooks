\documentclass[12pt]{article}
\usepackage[utf8]{inputenc}
\usepackage[english]{babel}
\usepackage{graphicx}
\usepackage{float}
\usepackage{amsmath}
\usepackage{xcolor}
\usepackage{enumitem}
\usepackage{titlesec}
\usepackage{amssymb}

\title{Tarea 2}
\author{Leonard David Vivas Dallos \\ Tomás Escobar Rivera \\ Grupo 13}
\date{\today}

\begin{document}

\maketitle

\tableofcontents

\section{Ejercicio 28 Sección 4.4}

Pruebe que si $A$ tiene un punto fijo no trivial (esto es, $Ax = x \neq 0$), luego $\| A \| \geq 1$ para cualquier norma de matriz subordinada.

\subsection{Solución}

Supongamos que $A$ tiene un punto fijo no trivial $x \neq 0$ tal que $Ax = x$. Consideremos cualquier norma matricial subordinada $||A||_s$. Veamos que $||A||_s \geq 1$. Por definición,
\begin{equation*}
    ||A||_s = \max\{||Ax||_s : \quad ||x||_s, x \neq 0\} = \sup_{x\neq0} \frac{||Ax||}{||x||}
\end{equation*}

Ahora, veamos que $||A||_s \geq 1$. Usando la hipótesis, estimemos $||A||_s$
\begin{equation*}
    \frac{||Ax||_s}{||x||_s} = \frac{||x||_s}{||x||_s} \quad \text{pues } Ax = x
\end{equation*}

Como $x\neq 0$, $||x||_s > 0$. Luego,
\begin{equation*}
    \frac{||Ax||_s}{||x||_s} = 1
\end{equation*}

Luego, para esta $x$ específica, el radio $\frac{||Ax||_s}{||x||_s}$ es $1$. Como estamos mirando el valor máximo de este radio sobre todo vector $x\neq 0$, se sigue que el máximo valor es al menos $1$. Por tanto,
\begin{equation*}
    ||A||_s \geq 1
\end{equation*}

\section{Ejercicio 1 Sección 4.6}

Pruebe que si $A$ es diagonalmente dominante y si $Q$ es escogido como en el método de Jacobi, entonces
\begin{equation*}
    \rho (I - Q^{-1}A) < 1
\end{equation*}

\subsection{Solución}

Como $A$ es diagonalmente dominante,
\begin{equation}
    |a_{ii}| > \sum_{\substack{j=1 \\ j\neq i}}^n |a_{ij}| \quad \text{para } i=1, 2, \ldots, n
\end{equation}

Supongamos que $Q$ es escogido como en el método de Jacobi, es decir, supongamos que $Q$ es la matriz conformada por la diagonal de $A$ (resto de entradas fuera de la diagonal son $0$). Veamos que
\begin{equation*}
    \rho (I - Q^{-1}A) < 1
\end{equation*}

Recordemos $\rho(B) = \underset{1 \leq i \leq n}{\max} |\sigma_i|$ con $\sigma_i$ valor propio de $B$, denominado el radio espectral de $B$. Además, tenemos que
\begin{equation}
    \rho(B) = \underset{||\cdot||}{\inf} ||B||
\end{equation}

De $(1)$ tenemos
\begin{flalign*}
    |a_{ii}| &> \sum_{\substack{j=1 \\ j\neq i}}^n |a_{ij}| \quad 1 \leq i \leq n \\
    &\therefore \sum_{\substack{j=1 \\ j\neq i}}^n \frac{|a_{ij}|}{|a_{ii}|} < 1 \quad 1 \leq i \leq n \\
    &\therefore \underset{1 \leq i \leq n}{\max} \sum_{\substack{j=1 \\ j\neq i}}^n \frac{|a_{ij}|}{|a_{ii}|} < 1 \\
    &\therefore ||I-Q^{-1}A||_\infty < 1
\end{flalign*}

Luego, de $(2)$ tenemos, por propiedades del ínfimo, que
\begin{equation*}
    \rho(I-Q^{-1}A) \leq ||I-Q^{-1}A||_\infty < 1
\end{equation*}
Por tanto, $\rho(I-Q^{-1}A) < 1$

\section{Ejercicio 11 Sección 6.1}

Pruebe que para cualquier polinomio $q$ de grado $\leq n-1$,
\begin{equation*}
    \sum_{i=0}^n q(x_i) \prod_{\substack{j=0 \\ j\neq i}}^n (x_i - x_j)^{-1}=0
\end{equation*}

\subsection{Solución}

Sea
\begin{equation*}
    p(x) = \sum_{i=0}^n q(x_i) \prod_{\substack{j=0 \\ j\neq i}}^n (x - x_j)
\end{equation*}

Notemos que $p(x)$ es de grado a lo más $n$. Veamos que $p(x)$ tiene $n+1$ raíces, $x_0, x_1, \ldots, x_n$. Notemos que para cualquier $x_i$
\begin{equation*}
    p(x_i) = \sum_{j=0}^n q(x_j) \prod_{\substack{k=0 \\ k\neq j}}^n (x_i - x_k)
\end{equation*}

Tenemos dos casos, cuando $j = i$, el término en la suma es
\begin{equation*}
    q(x_i) \prod_{\substack{k=0 \\ k\neq j}}^n (x_i - x_k)
\end{equation*}
y cuando $j \neq i$ tenemos
\begin{equation*}
    q(x_i) \prod_{\substack{k=0 \\ k\neq j}}^n (x_i - x_k) = 0 \quad (x_i - x_j = 0)
\end{equation*}

Luego, la suma tiene sólo un término distinto de $0$,
\begin{equation*}
    p(x_i) = q(x_i) \prod_{\substack{k=0 \\ k\neq j}}^n (x_i - x_k)
\end{equation*}

Como $q(x_i)$ es de grado $\leq n-1$, tiene a lo más $n-1$ raíces. Luego, hay al menos un $x_i$ tal que $q(x_i) \neq 0$. Para este $x_i$, $p(x_i) \neq 0$ y por tanto $p(x)$ tiene $n+1$ raíces.

Veamos que $p(x) = 0$. Como $p(x)$ es de grado $\leq 0$ con $n+1$ raíces, el polinomio debe ser $0$.
\begin{equation*}
    0 = p(x) = \sum_{i=0}^n q(x_i) \prod_{\substack{j=0 \\ j\neq i}}^n (x - x_j)
\end{equation*}

Dividiendo por el producto de diferencias obtenemos la igualdad deseada.

\section{Ejercicio 4 Sección 6.2}

Pruebe que si $f$ es un polinomio de grado $k$, entonces para $n > k$,
\begin{equation*}
    f[x_0,x_1, \ldots, x_n] = 0
\end{equation*}

\subsection{Solución}

Sea $p(x)$ un polinomio de interpolación de grado $\leq n$ para $f$, luego
\begin{equation}
    p(x_i)=f(x_i), \quad i=0, 1, \ldots, n
\end{equation}
Sea $q(x) = p(x) - f(x)$ un polinomio de grado $\leq n$. De $(3)$, sabemos que $q(x)$ tiene al menos $n+1$ raíces. Luego, $q(x)$ solo puede ser el polinomio $0$
\begin{flalign*}
    q(x) &= 0 \\
    \Rightarrow p(x) - f(x) &= 0 \\
    \Leftrightarrow p(x) &= f(x)
\end{flalign*}
Esto, por hipótesis implica que $p(x)$ es un polinomio de grado $k$. Ahora, por diferencias divididas tenemos
\begin{equation*}
    p(x) = \sum_{i=0}^n f[x_0,x_1, \ldots, x_n] \prod_{j=0}^{i-1} (x - x_j)
\end{equation*}
Luego, $f[x_0,x_1, \ldots, x_n]$ es el coeficiente de $x^n$. Por tanto,
\begin{equation*}
    f[x_0,x_1, \ldots, x_n] = 0 \quad \text{cuando} n > k
\end{equation*}

\end{document}
